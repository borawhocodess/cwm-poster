% Gemini theme
% https://github.com/anishathalye/gemini

\documentclass[final]{beamer}

% ====================
% Packages
% ====================

\usepackage[T1]{fontenc}
\usepackage{lmodern}
% \usepackage[size=custom,width=118.9,height=84.1,scale=1.0]{beamerposter}
\usepackage[size=custom,width=84.1,height=118.9,scale=1.0]{beamerposter}


% \makeatletter
% \def\input@path{{colorthemes/}}
% \makeatother

\usetheme{gemini}
\usecolortheme{itu}
\usepackage{graphicx}
\usepackage{booktabs}
\usepackage{tikz}
\usepackage{pgfplots}
\pgfplotsset{compat=1.14}
\usepackage{anyfontsize}

% ====================
% Lengths
% ====================

% If you have N columns, choose \sepwidth and \colwidth such that
% (N+1)*\sepwidth + N*\colwidth = \paperwidth
\newlength{\sepwidth}
\newlength{\colwidth}
\setlength{\sepwidth}{0.025\paperwidth}
\setlength{\colwidth}{0.3\paperwidth}

\newcommand{\separatorcolumn}{\begin{column}{\sepwidth}\end{column}}

% ====================
% Title
% ====================

\title{
% \hspace*{\sepwidth}\RaggedRight
% CReP:
Causal Representation Learning
% \newline\hspace*{\sepwidth}
for Time-Series Forecasting
}

\author{Salih \underline{Bora} Öztürk \inst{1} \and Hanne Raum \inst{1}}

\institute[shortinst]{\inst{1} University of Freiburg \samelineand}

% ====================
% Footer (optional)
% ====================

\footercontent{
  Causal World Models Seminar Poster Day 2026
  \hfill
  \href{mailto:oeztuers@cs.uni-freiburg.de}{oeztuers@cs.uni-freiburg.de}
}
% (can be left out to remove footer)

% ====================
% Logo (optional)
% ====================

% use this to include logos on the left and/or right side of the header:
\logoright{\includegraphics[height=2.22cm]{img/logoufr.png}}
% \logoleft{\includegraphics[height=7cm]{logo2.pdf}}

% ====================
% Body
% ====================

\begin{document}

\begin{frame}[t]
\begin{columns}[t]
\separatorcolumn

\begin{column}{\colwidth}

  \begin{alertblock}{TL;DR}

    \textbf{CReP}: Causal-oriented Representation Learning Predictor

    is a self-supervised framework that jointly performs multi-step time-series forecasting and causal analysis by decomposing high-dimensional observations into orthogonal cause, effect, and non-causal representations for a chosen target variable.

  \end{alertblock}

  \begin{block}{Introduction}

    \heading{Importance of the paper}

    \heading{Problem}

    \heading{Objectives}

  \end{block}

  \begin{block}{Method}

    \begin{figure}
      \centering
      \includegraphics[width=0.69\linewidth]{img/crepsupfig1.png}
      \caption{The flowchart of CReP method.}
    \end{figure}

    The framework is characterized by three key features:
    \begin{enumerate}
      \item Dynamic causation detection with the STI transformation mechanism (Fig. \ref{fig:crepfig1}a).
      \item Causal-oriented representation learning for multi-step predictions through the CReP (Fig. \ref{fig:crepfig1}b).
      \item Causal analysis of the target variable via $\alpha\beta$-LRP (Fig. \ref{fig:crepfig1}c).
    \end{enumerate}

    \begin{figure}
      \centering
      \includegraphics[width=0.69\linewidth]{img/crepfig1.png}
      \caption{Schematic illustration of the CReP framework.}
      \label{fig:crepfig1}
    \end{figure}

  \end{block}

  \begin{exampleblock}{Background Theory}

  \end{exampleblock}

\end{column}

\separatorcolumn

\begin{column}{\colwidth}

  \begin{block}{Results}

    \begin{figure}
      \centering
      \begin{minipage}[t]{0.31\linewidth}
        \centering
        \includegraphics[width=\linewidth]{img/crepfig2.png}
      \end{minipage}\hfill
      \begin{minipage}[t]{0.31\linewidth}
        \centering
        \includegraphics[width=\linewidth]{img/crepfig3.png}
      \end{minipage}\hfill
      \begin{minipage}[t]{0.31\linewidth}
        \centering
        \includegraphics[width=\linewidth]{img/crepfig4.png}
      \end{minipage}
      \caption{Performance of the CReP on the three simulation models.}
    \end{figure}

    \begin{figure}
      \centering
      \includegraphics[width=0.69\linewidth]{img/crepfig5.png}
      \caption{Performance of the CReP on the two real-world datasets.}
    \end{figure}

    \begin{figure}
      \centering
      \includegraphics[width=0.69\linewidth]{img/crepsupfig4.png}
      \caption{The normalized causal results of CReP on simulation datasets. Blue bars denote the top 10 variables by causal strength, purple bars highlight true causes or effects among them, and red bars indicate true causes or effects that were missed.}
    \end{figure}

  \end{block}
\end{column}

\separatorcolumn

\begin{column}{\colwidth}

  \begin{block}{Loss}

    \[
      \mathcal{L}=\lambda_1\mathcal{L}_{DS}+\lambda_2\mathcal{L}_{FC}+\lambda_3\mathcal{L}_{REC}+\lambda_4\mathcal{L}_{ORTH}
    \]

    \begin{itemize}
      \item \textbf{Determined-State Loss}: RMSE on known historical $y$.
      \item \textbf{Future-Consistency Loss}: RMSE between overlapping future estimates.
      \item \textbf{Reconstruction Loss}: assesses information recovery
        \begin{itemize}
          \item $\mathcal{L}_{REC\_X}$: spatiotemporal information $X$ from $(S,Z,V)$
          \item $\mathcal{L}_{REC\_S}$: latent cause representation $S$ recovered by $Y$
        \end{itemize}
      \item \textbf{Orthogonality Loss}: enforces orthogonality among $S$, $Z$, and $V$.
    \end{itemize}

    \heading{Ablation Study}

    Full loss achieves best overall performance; removing any term reduces causal identification reliability (accuracy/F1/recall), and often worsens forecasting RMSE.

  \end{block}

  \begin{block}{Supplementary Details}

    \begin{figure}
      \centering
      \includegraphics[width=0.69\linewidth]{img/crepsupfig2.png}
      \caption{Implementation details of CReP framework.}
    \end{figure}

    \begin{figure}
      \centering
      \includegraphics[width=0.69\linewidth]{img/crepsupfig3.png}
      \caption{The causal analysis process through interpretation method $\alpha\beta$-LRP.}
    \end{figure}

  \end{block}

  \begin{block}{Conclusion}

    \heading{Key Points}

    \heading{Future Work}

  \end{block}

  \begin{block}{References}

    \nocite{*}
    \footnotesize{\bibliographystyle{plain}\bibliography{poster}}

  \end{block}

\end{column}

\separatorcolumn
\end{columns}
\end{frame}

\end{document}
