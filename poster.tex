% Gemini theme
% https://github.com/anishathalye/gemini

\documentclass[final]{beamer}

% ====================
% Packages
% ====================

\usepackage[T1]{fontenc}
\usepackage{lmodern}
% \usepackage[size=custom,width=118.9,height=84.1,scale=1.0]{beamerposter}
\usepackage[size=custom,width=84.1,height=118.9,scale=1.0]{beamerposter}


% \makeatletter
% \def\input@path{{colorthemes/}}
% \makeatother

\usetheme{gemini}
\usecolortheme{itu}
\usepackage{graphicx}
\usepackage{booktabs}
\usepackage{tikz}
\usepackage{pgfplots}
\pgfplotsset{compat=1.14}
\usepackage{anyfontsize}

% ====================
% Lengths
% ====================

% If you have N columns, choose \sepwidth and \colwidth such that
% (N+1)*\sepwidth + N*\colwidth = \paperwidth
\newlength{\sepwidth}
\newlength{\colwidth}
\setlength{\sepwidth}{0.025\paperwidth}
\setlength{\colwidth}{0.45\paperwidth}

\newcommand{\separatorcolumn}{\begin{column}{\sepwidth}\end{column}}

% ====================
% Title
% ====================

\title{
% \hspace*{\sepwidth}\RaggedRight
Causal-oriented representation learning for time-series forecasting \\
% \newline\hspace*{\sepwidth}
based on the spatiotemporal information transformation
}

\author{Sihua Cai \inst{1,3} \and Hao Peng \inst{1,2,3} \and Rui Liu \inst{1} \and Pei Chen \inst{1}}

\institute[shortinst]{\inst{1} School of Mathematics, South China University of Technology \samelineand \inst{2} School of Future Technology, South China University of Technology \samelineand \inst{3} These authors contributed equally.}

% ====================
% Footer (optional)
% ====================

\footercontent{
  Causal World Models Seminar Poster Day 2026
  \hfill
  \href{mailto:oeztuers@cs.uni-freiburg.de}{oeztuers@cs.uni-freiburg.de}
}
% (can be left out to remove footer)

% ====================
% Logo (optional)
% ====================

% use this to include logos on the left and/or right side of the header:
% \logoright{\includegraphics[height=2.22cm]{img/logoufr.png}}
% \logoleft{\includegraphics[height=7cm]{logo2.pdf}}

% ====================
% Body
% ====================

\begin{document}

\begin{frame}[t]
\begin{columns}[t]
\separatorcolumn

\begin{column}{\colwidth}

  \begin{alertblock}{TL;DR}

    \textbf{CReP}: Causal-oriented Representation Learning Predictor

    is a self-supervised framework that jointly performs multi-step time-series forecasting and causal analysis by decomposing high-dimensional observations into orthogonal cause, effect, and non-causal representations for a chosen target variable.

  \end{alertblock}

  \begin{block}{Introduction}

    \textbf{Importance of the paper}

    \begin{itemize}
      \item Enables \textbf{simultaneous forecasting and causal analysis} in high-dimensional time-series systems.
      \item Provides \textbf{interpretable causal insights} (causes vs.\ effects) rather than black-box predictions.
    \end{itemize}

    \textbf{Problem}

      \begin{itemize}
        \item Existing approaches often forecast well but remain non-causal, or infer causality but are separate from forecasting.
      \end{itemize}

    \textbf{Objectives}

    % TODO: just describe it with words and put the definition of the terms to where it is needed to the loss function

    \begin{itemize}
    \item Given observations $X_t$ and a chosen target $y=x_k$:
      \begin{itemize}
        \item \textbf{Forecast} $y$ for multiple future steps in one shot.
        \item \textbf{Identify} which variables likely \textbf{cause} $y$ and which are likely \textbf{effects} of $y$.
      \end{itemize}
    \item Achieve this \textbf{self-supervised}.
  \end{itemize}

  \end{block}

  \begin{exampleblock}{Background Theory}

    \heading{Dynamic causation (reconstruction principle)}
      If variable $a$ drives $b$ in a dynamical system, then information about $a$ is embedded in the time series of $b$.
      Using delay embeddings there exists an implicit mapping (Takens-style):
      \[
        A_t=(a_t,\dots,a_{t+L-1})^\top,\quad
        B_t=(b_t,\dots,b_{t+L})^\top,\quad
        A_t \approx h(B_t)
      \]

    \heading{STI transformation (delay $\leftrightarrow$ non-delay)}
      Spatiotemporal information transformation allows a learned non-delay embedding $C_t$ to be topologically conjugate to $B_t$.
      \[
        A_t \approx \tilde{h}(C_t)
      \]
      This motivates learning a latent $Z_t$ that plays the role of a non-delay embedding from which $Y_t$ can be predicted:
      \[
        Y_t \approx g(Z_t)
      \]

  \end{exampleblock}

  \begin{block}{Method}

    \begin{figure}
      \centering
      \includegraphics[width=0.69\linewidth]{img/crepsupfig1.png}
      \caption{The flowchart of CReP method.}
    \end{figure}

    The framework is characterized by three key features:
    \begin{enumerate}
      \item Dynamic causation detection with the STI transformation mechanism (Fig. \ref{fig:crepfig1}a).
      \item Causal-oriented representation learning for multi-step predictions through the CReP (Fig. \ref{fig:crepfig1}b).
      \item Causal analysis of the target variable via $\alpha\beta$-LRP (Fig. \ref{fig:crepfig1}c).
    \end{enumerate}

    \begin{figure}
      \centering
      \includegraphics[width=0.69\linewidth]{img/crepfig1.png}
      \caption{Schematic illustration of the CReP framework.}
      \label{fig:crepfig1}
    \end{figure}

    % TODO: decide to write or not

    CReP uses an autoencoder to decompose inputs into cause, effect, and non-causal representations, and TCNs to perform stable multi-step forecasting (Fig.~\ref{fig:crepsupfig2}). The learned abstract representations are interpreted using $\alpha\beta$-LRP to map causal relevance back to the original variables and identify causes and effects (Fig.~\ref{fig:crepsupfig3}).

    % TODO: mention self-supervised

    \heading{Loss}

    \[
      \mathcal{L}=\lambda_1\mathcal{L}_{DS}+\lambda_2\mathcal{L}_{FC}+\lambda_3\mathcal{L}_{REC}+\lambda_4\mathcal{L}_{ORTH}
    \]

    \begin{itemize}
      \item \textbf{Determined-State Loss}: RMSE on known historical $y$.
      \item \textbf{Future-Consistency Loss}: RMSE between overlapping future estimates.
      \item \textbf{Reconstruction Loss}: assesses information recovery
        \begin{itemize}
          \item $\mathcal{L}_{REC\_X}$: spatiotemporal information $X$ from $(S,Z,V)$
          \item $\mathcal{L}_{REC\_S}$: latent cause representation $S$ recovered by $Y$
        \end{itemize}
      \item \textbf{Orthogonality Loss}: enforces orthogonality among $S$, $Z$, and $V$.
    \end{itemize}

    % TODO: ablation orally

    \heading{Ablation Study}

    Full loss achieves best overall performance; removing any term reduces causal identification reliability (accuracy/F1/recall), and often worsens forecasting RMSE.

  \end{block}

\end{column}

\separatorcolumn

\begin{column}{\colwidth}

  \begin{block}{Results}

    \begin{itemize}
      \item Simulation systems:
        \begin{itemize}
          \item Lorenz 96 (50x60 - 15) (Fig. \ref{fig:crepfig234} left)
          \item Kuramoto power grid (30x120 - 9) (Fig. \ref{fig:crepfig234} middle)
          \item Gene regulatory network (Dream4) (40x50 - 8) (Fig. \ref{fig:crepfig234} right)
        \end{itemize}
      \item Real datasets:
        \begin{itemize}
          \item Hong Kong cardiovascular inpatients (70x14 - 25) (Fig. \ref{fig:crepfig5})
          \item Japan COVID-19 transmission (30x47 - 14) (Fig. \ref{fig:crepfig5})
        \end{itemize}
      \item Metrics:
        \begin{itemize}
          \item RMSE (lower is better)
          \item PCC (higher is better)
        \end{itemize}
    \end{itemize}

    % TODO: select one

    \begin{figure}
      \centering
      \begin{minipage}[t]{0.31\linewidth}
        \centering
        \includegraphics[width=\linewidth]{img/crepfig2.png}
      \end{minipage}\hfill
      \begin{minipage}[t]{0.31\linewidth}
        \centering
        \includegraphics[width=\linewidth]{img/crepfig3.png}
      \end{minipage}\hfill
      \begin{minipage}[t]{0.31\linewidth}
        \centering
        \includegraphics[width=\linewidth]{img/crepfig4.png}
      \end{minipage}
      \caption{Performance of the CReP on the three simulation models.}
      \label{fig:crepfig234}
    \end{figure}

    \begin{figure}
      \centering
      \includegraphics[width=0.69\linewidth]{img/crepfig5.png}
      \caption{Performance of the CReP on the two real-world datasets.}
      \label{fig:crepfig5}
    \end{figure}

    \begin{table}
      \centering
      \begin{tabular}{l r r r r r r}
        \toprule
        & \multicolumn{2}{c}{Lorenz 96} & \multicolumn{2}{c}{Power grid} & \multicolumn{2}{c}{Dream4} \\
        \cmidrule(lr){2-3} \cmidrule(lr){4-5} \cmidrule(lr){6-7}
        \textbf{Method} & \textbf{RMSE} & \textbf{PCC} & \textbf{RMSE} & \textbf{PCC} & \textbf{RMSE} & \textbf{PCC} \\
        \midrule
        CReP     & \textbf{0.1086} & \textbf{0.9908} & \textbf{0.1127} & \textbf{0.9868} & \textbf{0.0984} & 0.8894 \\
        ARNN     & 0.2328          & 0.8206          & 0.1501          & 0.9784          & 0.2281          & 0.8167 \\
        STICM    & 0.1822          & 0.9752          & 0.1530          & 0.9854          & 0.1080          & \textbf{0.9928} \\
        LSTM     & 0.4883          & 0.8303          & 0.3916          & 0.9659          & 0.8063          & 0.0615 \\
        ARIMA    & 0.4204          & 0.6156          & 0.2558          & 0.9583          & 0.2650          & 0.1218 \\
        SVR      & 0.8311          & 0.1618          & 0.4856          & 0.9540          & 0.6779          & 0.0566 \\
        Informer & 0.1543          & 0.9005          & 0.1163          & 0.9837          & 0.4716          & 0.0489 \\
        \bottomrule
      \end{tabular}
      \caption{Comparison to other forecasting methods.}
      \label{tab:crepsuptab4}
    \end{table}

    % TODO: find out how did they measure this strength

    \begin{figure}
      \centering
      \includegraphics[width=0.69\linewidth]{img/crepsupfig4.png}
      \caption{The normalized causal results of CReP on simulation datasets. Blue bars denote the top 10 variables by causal strength, purple bars highlight true causes or effects among them, and red bars indicate true causes or effects that were missed.}
      \label{fig:crepsupfig4}
    \end{figure}

    % TODO: discussion orally

    \textbf{Critical discussion.}

    While CReP correctly ranks many true causes and effects among the top variables, Fig.~\ref{fig:crepsupfig4} also reveals \textbf{false positives} and \textbf{missed true causes}. Moreover, the forecasting comparison in Table~\ref{tab:crepsuptab4} is limited to relatively \textbf{basic baselines} (e.g., ARIMA, LSTM, SVR) and does not include more recent state-of-the-art foundation time-series models. In addition, history length choice in datasets, prediction length, and hyperparameter choices differ between datasets, raising questions about the fairness of the evaluations. A more controlled evaluation with matched forecasting horizons and stronger baselines would be required to fully assess CReP's performance.

  \end{block}

  \begin{block}{Conclusion}

    \heading{Key Points}
    \begin{itemize}
      \item Causes leave \textbf{detectable traces} in the dynamics of what they influence.
      \item CReP learns to \textbf{extract these traces} from high-dimensional time series.
      \item These representations are used to \textbf{predict the future} and \textbf{identify causes and effects}.
    \end{itemize}

    \heading{Future Work}
    \begin{itemize}
      \item Reducing false positives in causal discovery.
      \item Integrating different causal learning methods to enhance applicability across diverse domains.
      \item Exploring causal detection methods through active intervention rather than mere passive observations.
    \end{itemize}

  \end{block}

  \begin{block}{References}

    \nocite{*}
    \footnotesize{\bibliographystyle{plain}\bibliography{poster}}

  \end{block}

\end{column}

\separatorcolumn
\end{columns}
\end{frame}

\end{document}
